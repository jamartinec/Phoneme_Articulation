% Options for packages loaded elsewhere
% Options for packages loaded elsewhere
\PassOptionsToPackage{unicode}{hyperref}
\PassOptionsToPackage{hyphens}{url}
\PassOptionsToPackage{dvipsnames,svgnames,x11names}{xcolor}
%
\documentclass[
  letterpaper,
  DIV=11,
  numbers=noendperiod]{scrartcl}
\usepackage{xcolor}
\usepackage{amsmath,amssymb}
\setcounter{secnumdepth}{-\maxdimen} % remove section numbering
\usepackage{iftex}
\ifPDFTeX
  \usepackage[T1]{fontenc}
  \usepackage[utf8]{inputenc}
  \usepackage{textcomp} % provide euro and other symbols
\else % if luatex or xetex
  \usepackage{unicode-math} % this also loads fontspec
  \defaultfontfeatures{Scale=MatchLowercase}
  \defaultfontfeatures[\rmfamily]{Ligatures=TeX,Scale=1}
\fi
\usepackage{lmodern}
\ifPDFTeX\else
  % xetex/luatex font selection
\fi
% Use upquote if available, for straight quotes in verbatim environments
\IfFileExists{upquote.sty}{\usepackage{upquote}}{}
\IfFileExists{microtype.sty}{% use microtype if available
  \usepackage[]{microtype}
  \UseMicrotypeSet[protrusion]{basicmath} % disable protrusion for tt fonts
}{}
\makeatletter
\@ifundefined{KOMAClassName}{% if non-KOMA class
  \IfFileExists{parskip.sty}{%
    \usepackage{parskip}
  }{% else
    \setlength{\parindent}{0pt}
    \setlength{\parskip}{6pt plus 2pt minus 1pt}}
}{% if KOMA class
  \KOMAoptions{parskip=half}}
\makeatother
% Make \paragraph and \subparagraph free-standing
\makeatletter
\ifx\paragraph\undefined\else
  \let\oldparagraph\paragraph
  \renewcommand{\paragraph}{
    \@ifstar
      \xxxParagraphStar
      \xxxParagraphNoStar
  }
  \newcommand{\xxxParagraphStar}[1]{\oldparagraph*{#1}\mbox{}}
  \newcommand{\xxxParagraphNoStar}[1]{\oldparagraph{#1}\mbox{}}
\fi
\ifx\subparagraph\undefined\else
  \let\oldsubparagraph\subparagraph
  \renewcommand{\subparagraph}{
    \@ifstar
      \xxxSubParagraphStar
      \xxxSubParagraphNoStar
  }
  \newcommand{\xxxSubParagraphStar}[1]{\oldsubparagraph*{#1}\mbox{}}
  \newcommand{\xxxSubParagraphNoStar}[1]{\oldsubparagraph{#1}\mbox{}}
\fi
\makeatother


\usepackage{longtable,booktabs,array}
\usepackage{calc} % for calculating minipage widths
% Correct order of tables after \paragraph or \subparagraph
\usepackage{etoolbox}
\makeatletter
\patchcmd\longtable{\par}{\if@noskipsec\mbox{}\fi\par}{}{}
\makeatother
% Allow footnotes in longtable head/foot
\IfFileExists{footnotehyper.sty}{\usepackage{footnotehyper}}{\usepackage{footnote}}
\makesavenoteenv{longtable}
\usepackage{graphicx}
\makeatletter
\newsavebox\pandoc@box
\newcommand*\pandocbounded[1]{% scales image to fit in text height/width
  \sbox\pandoc@box{#1}%
  \Gscale@div\@tempa{\textheight}{\dimexpr\ht\pandoc@box+\dp\pandoc@box\relax}%
  \Gscale@div\@tempb{\linewidth}{\wd\pandoc@box}%
  \ifdim\@tempb\p@<\@tempa\p@\let\@tempa\@tempb\fi% select the smaller of both
  \ifdim\@tempa\p@<\p@\scalebox{\@tempa}{\usebox\pandoc@box}%
  \else\usebox{\pandoc@box}%
  \fi%
}
% Set default figure placement to htbp
\def\fps@figure{htbp}
\makeatother





\setlength{\emergencystretch}{3em} % prevent overfull lines

\providecommand{\tightlist}{%
  \setlength{\itemsep}{0pt}\setlength{\parskip}{0pt}}



 


\KOMAoption{captions}{tableheading}
\makeatletter
\@ifpackageloaded{caption}{}{\usepackage{caption}}
\AtBeginDocument{%
\ifdefined\contentsname
  \renewcommand*\contentsname{Table of contents}
\else
  \newcommand\contentsname{Table of contents}
\fi
\ifdefined\listfigurename
  \renewcommand*\listfigurename{List of Figures}
\else
  \newcommand\listfigurename{List of Figures}
\fi
\ifdefined\listtablename
  \renewcommand*\listtablename{List of Tables}
\else
  \newcommand\listtablename{List of Tables}
\fi
\ifdefined\figurename
  \renewcommand*\figurename{Figure}
\else
  \newcommand\figurename{Figure}
\fi
\ifdefined\tablename
  \renewcommand*\tablename{Table}
\else
  \newcommand\tablename{Table}
\fi
}
\@ifpackageloaded{float}{}{\usepackage{float}}
\floatstyle{ruled}
\@ifundefined{c@chapter}{\newfloat{codelisting}{h}{lop}}{\newfloat{codelisting}{h}{lop}[chapter]}
\floatname{codelisting}{Listing}
\newcommand*\listoflistings{\listof{codelisting}{List of Listings}}
\makeatother
\makeatletter
\makeatother
\makeatletter
\@ifpackageloaded{caption}{}{\usepackage{caption}}
\@ifpackageloaded{subcaption}{}{\usepackage{subcaption}}
\makeatother
\usepackage{bookmark}
\IfFileExists{xurl.sty}{\usepackage{xurl}}{} % add URL line breaks if available
\urlstyle{same}
\hypersetup{
  pdftitle={Defining Age-Dependent Threshold Curves},
  colorlinks=true,
  linkcolor={blue},
  filecolor={Maroon},
  citecolor={Blue},
  urlcolor={Blue},
  pdfcreator={LaTeX via pandoc}}


\title{Defining Age-Dependent Threshold Curves}
\author{}
\date{}
\begin{document}
\maketitle


\subsection{\texorpdfstring{Background: Phoneme Goodness Score curves
{(1/2)}}{Background: Phoneme Goodness Score curves (1/2)}}\label{background-phoneme-goodness-score-curves-12}

\begin{itemize}
\item
  Our primary input is a set of likelihood values assigned by a
  Wav2Vec2-based acoustic model (PyPLLRComputer software), to each
  target phoneme in speech recordings obtained from typically developing
  children.
\item
  We define Phoneme Goodness Score as the geometric mean of the
  likelihood assigned to the target phoneme.
\item
  For each phoneme we model a \textbf{growth curve} of Phoneme Goodness
  Score vs.~age using a \textbf{Bayesian Beta regression}.
\end{itemize}

\begin{center}\rule{0.5\linewidth}{0.5pt}\end{center}

\subsection{\texorpdfstring{Background: Phoneme Goodness Score curves
{(2/2)}}{Background: Phoneme Goodness Score curves (2/2)}}\label{background-phoneme-goodness-score-curves-22}

\begin{figure}[H]

{\centering \includegraphics[width=0.6\linewidth,height=\textheight,keepaspectratio]{./pictures/beta_age_plot_R.png}

}

\caption{Growth curve for R. Posterior median, 50\% and 95\% centered
credible bands.}

\end{figure}%

\begin{center}\rule{0.5\linewidth}{0.5pt}\end{center}

\subsection{What does the growth curve
mean?}\label{what-does-the-growth-curve-mean}

\begin{itemize}
\item
  For most phonemes, the \textbf{goodness score increases with age}.
\item
  This increase might involve:

  \begin{itemize}
  \tightlist
  \item
    \textbf{Linguistic component}

    \begin{itemize}
    \tightlist
    \item
      improved articulation / acquisition
    \end{itemize}
  \item
    \textbf{Physiological component}

    \begin{itemize}
    \tightlist
    \item
      maturation / growth of the vocal tract, etc.
    \end{itemize}
  \end{itemize}
\end{itemize}

\begin{center}\rule{0.5\linewidth}{0.5pt}\end{center}

\subsection{\texorpdfstring{How can growth curves inform phoneme
articulation mastery?
{(1/3)}}{How can growth curves inform phoneme articulation mastery? (1/3)}}\label{how-can-growth-curves-inform-phoneme-articulation-mastery-13}

We aim to establish \textbf{age-dependent threshold curves} on the
Phoneme Goodness Score trajectories such that:

\begin{itemize}
\item
  For each age, the value in the threshold curve represents the
  \textbf{cutpoint} or minimum score consistent with \textbf{mastery of
  articulation}, and
\item
  Values \textbf{below} this cut-point indicate \textbf{insufficient
  mastery} or incomplete phonological acquisition.
\end{itemize}

\begin{center}\rule{0.5\linewidth}{0.5pt}\end{center}

\subsection{\texorpdfstring{How can growth curves inform phoneme
articulation mastery?
{(2/3)}}{How can growth curves inform phoneme articulation mastery? (2/3)}}\label{how-can-growth-curves-inform-phoneme-articulation-mastery-23}

\begin{figure}[H]

{\centering \includegraphics[width=0.55\linewidth,height=\textheight,keepaspectratio]{./pictures/beta_age_plot_R_corte1.png}

}

\caption{A hypothetical age-invariant mastery threshold.}

\end{figure}%

If a \textbf{constant (age-invariant) threshold line} were sufficient,
we could hypothesize that \textbf{the phoneme goodness score mainly
reflects articulation mastery}, with minimal influence from
physiological maturation.

\begin{center}\rule{0.5\linewidth}{0.5pt}\end{center}

\subsection{\texorpdfstring{How can growth curves inform phoneme
articulation mastery?
{(3/3)}}{How can growth curves inform phoneme articulation mastery? (3/3)}}\label{how-can-growth-curves-inform-phoneme-articulation-mastery-33}

\begin{figure}[H]

{\centering \includegraphics[width=0.55\linewidth,height=\textheight,keepaspectratio]{./pictures/beta_age_plot_R_corte2.png}

}

\caption{A hypothetical age-dependent threshold.}

\end{figure}%

If the required threshold \textbf{increases substantially with age},
then the goodness score is, to some extent, \textbf{reflecting
physiological maturation rather than pure articulatory mastery}.

\begin{center}\rule{0.5\linewidth}{0.5pt}\end{center}

\subsection{\texorpdfstring{Crowe \& McLeod: normative acquisition
curves.
{(1/2)}}{Crowe \& McLeod: normative acquisition curves. (1/2)}}\label{crowe-mcleod-normative-acquisition-curves.-12}

Crowe \& McLeod (2020) analyzed data from 15 studies (Approx. 18,900
children) and reported, for each English consonant:

\begin{itemize}
\item
  The \textbf{mean age} at which children reach\\
  \textbf{50\%, 75\%, and 90\% accuracy} in producing that phoneme, and
  the corresponding \textbf{standard deviations} across studies.
\item
  \(x\%\) criterion indicates that \(x\%\) of the participants produced
  the consonant correctly.
\end{itemize}

\begin{center}\rule{0.5\linewidth}{0.5pt}\end{center}

\subsection{\texorpdfstring{Crowe \& McLeod: normative acquisition
curves.
{(2/2)}}{Crowe \& McLeod: normative acquisition curves. (2/2)}}\label{crowe-mcleod-normative-acquisition-curves.-22}

\includegraphics[width=0.7\linewidth,height=\textheight,keepaspectratio]{./pictures/C&M.png}

We are using these normative values as \textbf{reference anchors} for
the cut-point definition.

\begin{center}\rule{0.5\linewidth}{0.5pt}\end{center}

\subsection{\texorpdfstring{AAPS: probability of success vs.~age
{(1/2)}}{AAPS: probability of success vs.~age (1/2)}}\label{aaps-probability-of-success-vs.-age-12}

\begin{itemize}
\item
  \textbf{Arizona Articulation and Phonology Scale (AAPS)}: standardized
  assessment that evaluates whether a child correctly produces a target
  phoneme in a structured elicitation task.
\item
  We use AAPS scoring data from the same participants for whom Phoneme
  Goodness values were computed.
\item
  From these binary (0/1) outcomes we fit a \textbf{Bayesian Binomial
  model} to estimate, for each phoneme, the \textbf{probability of
  correct production} as a function of age.
\end{itemize}

\begin{center}\rule{0.5\linewidth}{0.5pt}\end{center}

\subsection{\texorpdfstring{AAPS: probability of success vs.~age
{(2/2)}}{AAPS: probability of success vs.~age (2/2)}}\label{aaps-probability-of-success-vs.-age-22}

\begin{figure}[H]

{\centering \includegraphics[width=0.6\linewidth,height=\textheight,keepaspectratio]{./pictures/binomial_age_plot_R.png}

}

\caption{Growth curve for R. Posterior median, 50\% and 95\% centered
credible bands.}

\end{figure}%

\begin{center}\rule{0.5\linewidth}{0.5pt}\end{center}

\subsection{Integrating normative and model-based
information}\label{integrating-normative-and-model-based-information}

Our goal is to define, for each phoneme and age, a \textbf{Phoneme
Goodness threshold} indicating mastery of articulation.

We integrate three components:

\begin{enumerate}
\def\labelenumi{\arabic{enumi}.}
\item
  \textbf{Crowe \& McLeod (2020):} Normative ages for \textbf{50\%,
  75\%, and 90\% accuracy}.
\item
  \textbf{AAPS Bayesian Binomial model:} For each age,\\
  \(\color{blue}{q(\text{age})} = P(\text{100% correct} \mid \text{age})
  \) gives the \emph{probability of mastery}.
\item
  \textbf{Goodness score (Y) - Bayesian Beta model:} We find
  \(\color{red}{x_q(\text{age})}\) such that
  \(P\big(Y \ge \color{red}{x_q(\text{age})} \mid \text{age}\big)
    = \color{blue}{q(\text{age})}\).
\end{enumerate}

\(\color{red}{x_q(\text{age})}\) is the proposed \textbf{age-dependent
Phoneme Goodness Score Threshold}.

\begin{center}\rule{0.5\linewidth}{0.5pt}\end{center}

\subsection{Phoneme Goodness
Threshold}\label{phoneme-goodness-threshold}

\begin{figure}[H]

{\centering \includegraphics[width=0.8\linewidth,height=\textheight,keepaspectratio]{./pictures/R_age_plot5_cuts.png}

}

\end{figure}%

\begin{center}\rule{0.5\linewidth}{0.5pt}\end{center}

\subsection{Phoneme Goodness
Threshold}\label{phoneme-goodness-threshold-1}

\begin{figure}[H]

{\centering \includegraphics[width=0.8\linewidth,height=\textheight,keepaspectratio]{./pictures/R_age_plot5_cuts2.png}

}

\end{figure}%

\begin{center}\rule{0.5\linewidth}{0.5pt}\end{center}

\subsection{Phoneme Goodness
Threshold}\label{phoneme-goodness-threshold-2}

\begin{figure}[H]

{\centering \includegraphics[width=1\linewidth,height=\textheight,keepaspectratio]{./pictures/R_age_plot5_cuts3.png}

}

\end{figure}%

\begin{itemize}
\item
  \textbf{Red line:} Mean acquisition age = \textbf{35 months} (Crowe \&
  McLeod).\\
  Projecting the red segment onto the \textbf{x-axis} gives the
  \textbf{±2 SD age range} around this mean.\\
  Threshold at this age: \textbf{0.06} (Score ≥ 0.06 → likely mastered).
\item
  \textbf{Green line:} Threshold at \textbf{48 months} = \textbf{0.02}.
\item
  \textbf{Purple line:} Threshold at \textbf{67 months} = \textbf{0.02}.
\item
  \textbf{Black curve:} Connects these three values →
  \textbf{age-dependent Goodness Score threshold}.
\end{itemize}

\begin{center}\rule{0.5\linewidth}{0.5pt}\end{center}

\subsection{Phoneme Goodness
Threshold}\label{phoneme-goodness-threshold-3}

\begin{figure}[H]

{\centering \includegraphics[width=0.8\linewidth,height=\textheight,keepaspectratio]{./pictures/S_age_plot5_cuts.png}

}

\end{figure}%

\begin{center}\rule{0.5\linewidth}{0.5pt}\end{center}

\subsection{Phoneme Goodness
Threshold}\label{phoneme-goodness-threshold-4}

\begin{figure}[H]

{\centering \includegraphics[width=0.8\linewidth,height=\textheight,keepaspectratio]{./pictures/R_age_plot5_cuts.png}

}

\end{figure}%

\begin{center}\rule{0.5\linewidth}{0.5pt}\end{center}

\subsection{Phoneme Goodness
Threshold}\label{phoneme-goodness-threshold-5}

\begin{figure}[H]

{\centering \includegraphics[width=0.8\linewidth,height=\textheight,keepaspectratio]{./pictures/T_age_plot5_cuts.png}

}

\end{figure}%

\begin{center}\rule{0.5\linewidth}{0.5pt}\end{center}

\subsection{Phoneme Goodness
Threshold}\label{phoneme-goodness-threshold-6}

\begin{figure}[H]

{\centering \includegraphics[width=0.8\linewidth,height=\textheight,keepaspectratio]{./pictures/CH_age_plot5_cuts.png}

}

\end{figure}%

\begin{center}\rule{0.5\linewidth}{0.5pt}\end{center}

\subsection{Phoneme Goodness
Threshold}\label{phoneme-goodness-threshold-7}

\begin{figure}[H]

{\centering \includegraphics[width=0.8\linewidth,height=\textheight,keepaspectratio]{./pictures/JH_age_plot5_cuts.png}

}

\end{figure}%

\begin{center}\rule{0.5\linewidth}{0.5pt}\end{center}

\subsection{Phoneme Goodness
Threshold}\label{phoneme-goodness-threshold-8}

\begin{figure}[H]

{\centering \includegraphics[width=0.8\linewidth,height=\textheight,keepaspectratio]{./pictures/L_age_plot5_cuts.png}

}

\end{figure}%

\begin{center}\rule{0.5\linewidth}{0.5pt}\end{center}

\subsection{Phoneme Goodness
Threshold}\label{phoneme-goodness-threshold-9}

\begin{figure}[H]

{\centering \includegraphics[width=0.8\linewidth,height=\textheight,keepaspectratio]{./pictures/SH_age_plot5_cuts.png}

}

\end{figure}%

\begin{center}\rule{0.5\linewidth}{0.5pt}\end{center}

\subsection{Phoneme Goodness
Threshold}\label{phoneme-goodness-threshold-10}

\begin{figure}[H]

{\centering \includegraphics[width=0.8\linewidth,height=\textheight,keepaspectratio]{./pictures/Z_age_plot5_cuts.png}

}

\end{figure}%




\end{document}
